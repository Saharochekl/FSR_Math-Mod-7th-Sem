\documentclass[12pt,a4paper]{article}

\usepackage[T2A]{fontenc}
\usepackage[utf8]{inputenc}
\usepackage[russian]{babel}

\usepackage{geometry}
\geometry{margin=2.5cm}

\usepackage{amsmath, amssymb, amsthm}
\usepackage{graphicx}
\usepackage{float}
\usepackage{hyperref}
\hypersetup{hidelinks}

\usepackage{booktabs}

\newtheorem{theorem}{Теорема}
\newtheorem{remark}{Замечание}

\title{Метод штрафных функций для задач условной оптимизации}
\author{}
\date{}

\begin{document}
\maketitle

\section{Постановка задачи}

Рассматривается задача условной оптимизации вида
\begin{equation}\label{eq:main_problem}
\min_{x \in \mathbb{R}^n} \; J(x)
\quad \text{при условиях} \quad
g_i(x) \le 0,\; i=1,\dots,m,\qquad
h_j(x)=0,\; j=1,\dots,p,
\end{equation}
где $J(x)$ --- целевая функция, $g_i(x)$ --- неравенства, $h_j(x)$ --- равенства.
Обозначим допустимое множество
\[
\Omega = \{x \in \mathbb{R}^n : g_i(x)\le 0,\ i=1,\dots,m;\; h_j(x)=0,\ j=1,\dots,p\}.
\]

\section{Идея метода штрафных функций}

Метод штрафных функций сводит задачу \eqref{eq:main_problem} к последовательности задач безусловной оптимизации:
\begin{equation}\label{eq:penalized_family}
\min_{x \in \mathbb{R}^n} \; F(x,r) = J(x) + r\,\Phi(x),
\end{equation}
где $r>0$ --- параметр штрафа, а $\Phi(x)\ge 0$ --- штрафная функция, которая обращается в ноль на допустимом множестве и положительна вне его.

\subsection{Штраф для неравенств}

Для одного неравенства $g(x)\le 0$ типичный выбор:
\begin{equation}\label{eq:ineq_penalty}
\Phi(x) = \bigl(\max\{0,g(x)\}\bigr)^2.
\end{equation}
Тогда:
\[
g(x)\le 0 \;\Rightarrow\; \Phi(x)=0,\qquad
g(x)>0 \;\Rightarrow\; \Phi(x)=g(x)^2.
\]
Для нескольких неравенств обычно берут сумму:
\[
\Phi(x)=\sum_{i=1}^m \bigl(\max\{0,g_i(x)\}\bigr)^2.
\]

\subsection{Штраф для равенств}

Для равенства $h(x)=0$ стандартный выбор:
\begin{equation}\label{eq:eq_penalty}
\Phi(x)=h(x)^2,
\end{equation}
и, аналогично, для нескольких равенств берут сумму квадратов.

\section{Сходимость и связь с условиями оптимальности}

Метод штрафа в рассматриваемой реализации относится к \emph{внешним} штрафам: минимизация \eqref{eq:penalized_family} может проводиться как из допустимых, так и из недопустимых точек, а рост $r$ заставляет решения приближаться к $\Omega$.

\begin{remark}[Интуиция]
При фиксированном $r$ задача \eqref{eq:penalized_family} является обычной задачей безусловной оптимизации. При увеличении $r$ любое нарушение ограничений становится всё дороже, и минимум $F(x,r)$ смещается к границе/внутрь допустимой области. 
\end{remark}

Для задач с достаточно гладкими функциями при выполнении регулярности ограничений оптимальное решение $x^\star$ удовлетворяет условиям ККТ (Каруша--Куна--Таккера): существуют множители Лагранжа $\lambda_i\ge 0$ и $\mu_j$ такие, что
\begin{align}
\nabla J(x^\star) + \sum_{i=1}^m \lambda_i \nabla g_i(x^\star) + \sum_{j=1}^p \mu_j \nabla h_j(x^\star) &= 0, \label{eq:kkt_stationarity}\\
g_i(x^\star) \le 0,\quad h_j(x^\star)=0,\quad \lambda_i &\ge 0,\label{eq:kkt_feasibility}\\
\lambda_i g_i(x^\star) &= 0 \quad (i=1,\dots,m). \label{eq:kkt_complementarity}
\end{align}
Методы штрафа являются одним из способов приближённо находить решения, удовлетворяющие \eqref{eq:kkt_stationarity}--\eqref{eq:kkt_complementarity}, через последовательность безусловных задач.

\section{Алгоритм (как реализовано в коде)}

В коде используется следующая схема:
\begin{enumerate}
  \item Задаётся начальная точка $x^{(0)}$, точность $\varepsilon$, начальный штраф $r_0$.
  \item На $k$-й итерации решается безусловная задача:
  \[
  x^{(k+1)} \in \arg\min_{x} \; F(x,r_k)=J(x)+r_k\Phi(x).
  \]
  \item Проверяется критерий близости к ограничениям. В коде используется условие вида
  \[
  |g(x^{(k+1)})|<\varepsilon
  \]
  (для равенств аналогично).
  \item Если критерий не выполнен, штраф увеличивается (геометрический рост): $r_{k+1}=2r_k$.
\end{enumerate}

\begin{remark}[Практический смысл параметров]
Большие $r$ сильнее «прижимают» к ограничениям, но делают задачу плохо обусловленной: минимум $F(x,r)$ становится узкой «канавкой» вдоль допустимого множества, и численной оптимизации сложнее. Поэтому штраф наращивают постепенно.
\end{remark}

\section{Эксперименты и постановки}

Далее рассматриваются три конкретные задачи (ровно те, что реализованы).

\subsection{Задача 1: квадратичная функция и линейное неравенство}

\paragraph{Постановка.}
\[
\min\; J(x,y)=x^2+y^2
\quad \text{при} \quad
-x-y+1 \le 0 \;\;\; (\Leftrightarrow x+y \ge 1).
\]
Штраф:
\[
\Phi(x,y) = \bigl(\max\{0,\,-x-y+1\}\bigr)^2,
\quad
F(x,y;r)=x^2+y^2+r\,\Phi(x,y).
\]

\paragraph{Геометрическая интерпретация.}
Линия $x+y=1$ является границей допустимой области $x+y\ge 1$.
Минимизируется расстояние до начала координат, но точка $(0,0)$ недопустима, поэтому решение лежит на границе $x+y=1$ (минимальная норма при данном линейном ограничении).

\begin{figure}[H]
  \centering
  \includegraphics[width=0.82\textwidth]{penalty_method1.png}
  \caption{Траектория метода штрафа для задачи 1: уровни $x^2+y^2$ и ограничение $x+y\ge 1$.}
\end{figure}

\subsection{Задача 2: билинейная функция и ограничение окружностью}

\paragraph{Постановка.}
\[
\min\; J(x,y)=xy
\quad \text{при} \quad
x^2+y^2=25.
\]
В коде ограничение задано как
\[
g(x,y)=x^2+y^2-25,
\]
и применяется штраф квадрата (как для равенства, хотя в обозначениях использован $g$):
\[
\Phi(x,y)=g(x,y)^2=(x^2+y^2-25)^2,\quad
F(x,y;r)=xy+r(x^2+y^2-25)^2.
\]

\paragraph{Интуиция решения.}
На окружности радиуса $5$ величина $xy$ минимальна, когда $x$ и $y$ противоположных знаков и по модулю равны (точки близкие к направлению $(-1,1)$).
Это согласуется с параметризацией $x=5\cos\theta$, $y=5\sin\theta$, где $xy=\frac{25}{2}\sin(2\theta)$ минимально при $\sin(2\theta)=-1$.

\begin{figure}[H]
  \centering
  \includegraphics[width=0.82\textwidth]{penalty_method2.png}
  \caption{Метод штрафа для задачи 2: уровни $xy$ и ограничение $x^2+y^2=25$.}
\end{figure}

\subsection{Задача 3: негладкая долина и равенство $x=a$}

\paragraph{Постановка.}
\[
\min\; J(x,y)=x^2+(1-xy)^2
\quad \text{при} \quad
x=a.
\]
Ограничение записано как $g(x,y)=x-a$, а штраф (как для равенства):
\[
\Phi(x,y)=(x-a)^2,\quad
F(x,y;r)=x^2+(1-xy)^2 + r(x-a)^2.
\]

\paragraph{Комментарий.}
При фиксированном $x=a$ задача сводится к одномерной по $y$:
\[
J(a,y)=a^2+(1-a y)^2,
\]
и минимум по $y$ достигается при $ay=1$, то есть $y^\star = 1/a$.

\begin{figure}[H]
  \centering
  \includegraphics[width=0.82\textwidth]{penalty_method3.png}
  \caption{Метод штрафа для задачи 3: уровни $x^2+(1-xy)^2$ и ограничение $x=a$.}
\end{figure}

\section{Замечания к реализации}

\begin{enumerate}
  \item В задачах 1 и 2 используется «квадрат положительной части» $\bigl(\max(0,g)\bigr)^2$ для неравенств. Это создаёт негладкость по границе $g=0$ (из-за $\max$), но на практике для численных методов часто приемлемо.
  \item В задаче 2 ограничение является равенством $x^2+y^2=25$, однако код применяет структуру как в неравенствах, но затем проверяет $|g(x)|<\varepsilon$. Фактически это штраф для равенства, и это корректно.
  \item Остановка по условию $|g(x)|<\varepsilon$ является эвристикой. Более строгий критерий для неравенств: $\max(0,g(x))<\varepsilon$.
  \item Увеличение штрафа $r \leftarrow 2r$ даёт быстрый рост. Иногда полезно ограничивать $r$ сверху или адаптивно выбирать шаг роста, чтобы не «сломать» численную оптимизацию.
\end{enumerate}

\section{Вывод}

Метод штрафных функций превращает условную задачу в последовательность безусловных, добавляя штраф за нарушение ограничений. В экспериментах видно, что траектории итераций приближаются к допустимому множеству, а затем движутся вдоль него к точке минимума исходной задачи.

\end{document}