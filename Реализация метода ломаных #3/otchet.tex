\documentclass[12pt,a4paper]{article}
\usepackage[T2A]{fontenc}
\usepackage[utf8]{inputenc}
\usepackage[russian]{babel}
\usepackage{amsmath, amssymb}
\usepackage{geometry}
\usepackage{graphicx}
\usepackage{hyperref}
\usepackage{booktabs}

\usepackage{float}
\usepackage{subcaption}
\graphicspath{{out/plots/}}

\usepackage{algorithm}
\usepackage{algpseudocode}

\geometry{margin=2.5cm}
\hypersetup{hidelinks}

\title{Глобальная минимизация таблично заданных функций:\\
метод ломаных (Пиявского--Шуберта) и классические одномерные методы}
\author{}
\date{}

\begin{document}
\maketitle

\section{Постановка задачи}

Дана функция \(f\), заданная таблично наборами точек \(\{(x_i,y_i)\}_{i=1}^n\).
Требуется найти приближение точки минимума
\[
x^\ast \in [a,b], \qquad f(x^\ast)=\min_{x\in[a,b]} f(x),
\]
где \([a,b]\) --- границы области определения, извлечённые из табличных \(x_i\).

Поскольку исходные данные дискретны, в алгоритме строится \textbf{линейная интерполяция}
\(\tilde f(x)\), и далее минимизируется именно \(\tilde f\) на \([a,b]\).

\section{Предобработка табличных данных}

Входные файлы (\texttt{.txt} или \texttt{.csv}) содержат минимум два столбца:
\(\,x\) (первый столбец) и \(\,y\) (второй столбец).

\subsection{Разбор формата и очистка}
Алгоритм чтения данных:
\begin{itemize}
  \item выбирает разделитель \texttt{;} или \texttt{,} по первой подходящей строке;
  \item берёт \(\,y\) из 2-го столбца только если он распознаётся как число;
  \item пытается интерпретировать \(\,x\) как дату/время; если удаётся, переводит в секунды относительно первого времени;
  \item иначе пытается интерпретировать \(\,x\) как число; если не получается --- использует индекс строки;
  \item удаляет дубликаты по \(x\), сортирует по \(x\).
\end{itemize}

\subsection{Границы интервала}
После сортировки:
\[
a=\min_i x_i,\qquad b=\max_i x_i.
\]

\section{Линейная интерполяция}

Пусть \((x_1,y_1),\dots,(x_m,y_m)\) --- очищенные точки с \(x_1<\dots<x_m\).
Определим \(\tilde f(x)\) как кусочно-линейную интерполяцию:
для \(x\in[x_k,x_{k+1}]\)
\[
\tilde f(x)= y_k + \frac{y_{k+1}-y_k}{x_{k+1}-x_k}\,(x-x_k).
\]
Вне узлов интервалов \(\tilde f\) вычисляется стандартной операцией \texttt{interp}.

\paragraph{Замечание.}
После этого минимизируется не исходная (неизвестная) функция, а её аппроксимация \(\tilde f\).
Если таблица редкая или шумная, положение минимума может смещаться.

\section{Липшицевость и оценка константы}

\subsection{Определение}
Функция \(f\) называется липшицевой на \([a,b]\), если существует \(L>0\), такое что
\[
|f(x)-f(y)| \le L|x-y|,\qquad \forall x,y\in[a,b].
\]
\(L\) называют \textbf{константой Липшица}.

\subsection{Оценка \(L\) по таблице}
Для кусочно-линейной \(\tilde f\) достаточно оценить максимальный модуль наклона на отрезках:
\[
L_0=\max_{k=1,\dots,m-1}\left|\frac{y_{k+1}-y_k}{x_{k+1}-x_k}\right|.
\]
В реализации применяется запасной множитель \(r>1\):
\[
L = r\,L_0.
\]
Это делает метод устойчивее, если наклоны недооценены из-за дискретизации/шума.

\section{Метод ломаных (Пиявского--Шуберта)}

\subsection{Идея: нижняя оценка через конусы}
Пусть \(f\) липшицева с константой \(L\). Тогда из неравенства Липшица:
\[
f(x)\ge f(x_i) - L|x-x_i| \quad \forall x.
\]
Для набора уже вычисленных точек \(x_i\) можно построить \textbf{глобальный минорант}:
\[
g(x)=\max_i\Bigl(f(x_i)-L|x-x_i|\Bigr),
\]
который удовлетворяет \(g(x)\le f(x)\) для всех \(x\).
График \(g\) представляет собой верхнюю огибающую ``конусов'' и выглядит как ломаная.

\subsection{Характеристика интервала и точка следующего измерения}
Рассмотрим два соседних узла \(x_i < x_{i+1}\) и значения \(f_i=f(x_i)\), \(f_{i+1}=f(x_{i+1})\).
На интервале \([x_i,x_{i+1}]\) минимум минорната достигается в точке пересечения двух прямых:
\[
x_{\text{new}}=\frac{x_i+x_{i+1}}{2}-\frac{f_{i+1}-f_i}{2L}.
\]
Минимальное значение минорната на этом интервале (нижняя оценка на минимум \(f\)) равно
\[
R_i=\frac{f_i+f_{i+1}}{2}-\frac{L}{2}(x_{i+1}-x_i).
\]
Далее выбирают интервал с \textbf{наиболее перспективной} (наименьшей) нижней оценкой:
\[
i^\ast = \arg\min_i R_i,
\]
вычисляют \(f(x_{\text{new}})\), добавляют новую точку и повторяют процесс.

\subsection{Итерационная схема}
Инициализация: задать \([a,b]\), оценку \(L\), начальные узлы \(x_1=a\), \(x_2=b\), вычислить \(f(x_1)\), \(f(x_2)\).

\paragraph{Итерации.}
Пусть на шаге \(k\) имеется упорядоченный набор узлов \(x_1<\dots<x_m\) и значения \(f(x_j)\).

\begin{enumerate}
  \item Для каждого интервала \([x_i,x_{i+1}]\) вычислить характеристику
  \[
    R_i=\frac{f(x_i)+f(x_{i+1})}{2}-\frac{L}{2}(x_{i+1}-x_i).
  \]

  \item Найти индекс самого перспективного интервала:
  \[
    i^\ast = \arg\min_i R_i.
  \]

  \item Вычислить новую точку испытания:
  \[
    x_{\text{new}}=\frac{x_{i^\ast}+x_{i^\ast+1}}{2}-\frac{f(x_{i^\ast+1})-f(x_{i^\ast})}{2L},
  \]
  и ограничить её интервалом \([x_{i^\ast},x_{i^\ast+1}]\).

  \item Вычислить \(f(x_{\text{new}})\), добавить точку \(x_{\text{new}}\) в набор узлов (с сохранением сортировки).
\end{enumerate}

\paragraph{Критерии остановки.}
Итерации прекращают при выполнении хотя бы одного из условий:
\begin{itemize}
  \item \(\bigl|f^{(k)}_{\min}-f^{(k-1)}_{\min}\bigr|<\varepsilon\), где \(f^{(k)}_{\min}=\min_j f(x_j)\) на итерации \(k\);
  \item \(\bigl|x^{(k)}_{\min}-x^{(k-1)}_{\min}\bigr|<\delta\), где \(x^{(k)}_{\min}=\arg\min_j f(x_j)\) на итерации \(k\).
\end{itemize}



\subsection{Сходимость (интуитивно)}
Если \(L\) не занижен (то есть \(L\ge L_{\text{true}}\)), то минорнат \(g(x)\) корректен,
а последовательность уточнений сужает ``подозрительные'' интервалы.
При росте числа узлов метод стремится к глобальному минимуму (при стандартных предположениях).

\section{Классические одномерные методы (для унимодальных функций)}

Далее применяются три метода, которые гарантируют корректность при \textbf{унимодальности}
функции на \([a,b]\) (имеется единственный минимум и нет других локальных минимумов).

\subsection{Дихотомия}
На каждом шаге берутся точки
\[
x_1 = m-\delta,\quad x_2=m+\delta,\quad m=\frac{a+b}{2},
\]
и по сравнению \(f(x_1)\) и \(f(x_2)\) отбрасывается половина интервала.
Остановка при \(b-a\le \varepsilon\).

\subsection{Золотое сечение}
Используется свойство самоподобия разбиений при коэффициенте
\[
\varphi = \frac{\sqrt{5}-1}{2}.
\]
Сохраняется одна из ранее вычисленных точек, поэтому после старта требуется по 1 новому вычислению \(f\) за итерацию.

\subsection{Поиск Фибоначчи}
Число шагов заранее определяется через числа Фибоначчи \(F_n\), выбираемые так, чтобы
\[
F_n \gtrsim \frac{b-a}{\varepsilon}.
\]
Метод минимизирует число вычислений \(f\) при фиксированной точности и унимодальности.

\section{Формирование результатов}

Для каждого входного файла:
\begin{itemize}
  \item строится интерполяция \(\tilde f\);
  \item оценивается \(L\);
  \item выполняется \(N\) итераций метода ломаных, сохраняются точки \((x,f(x))\);
  \item выполняются дихотомия, золотое сечение и Фибоначчи на той же \(\tilde f\);
  \item сохраняются текстовый отчёт и сводная таблица.
\end{itemize}

Отдельно строятся графики:
\begin{itemize}
  \item \(\tilde f(x)\) на равномерной сетке;
  \item минорнат \(g(x)=\max_i(f(x_i)-L|x-x_i|)\) для текущего множества узлов метода ломаных.
\end{itemize}

\section{Графики (out/plots/)}

Ниже приведены графики, автоматически сохранённые скриптом в каталог \texttt{out/plots/}.
На каждом изображении показаны интерполированная функция \(f(x)\) и нижняя оценка \(g(x; x_i)=\max_i(f(x_i)-L|x-x_i|)\) для текущего набора узлов метода ломаных.

\subsection{angle-func}
\begin{figure}[H]
  \centering
  \begin{subfigure}{0.32\textwidth}
    \centering
    \includegraphics[width=\linewidth]{angle-func_iter1.png}
    \caption{Итерация 1}
  \end{subfigure}
  \begin{subfigure}{0.32\textwidth}
    \centering
    \includegraphics[width=\linewidth]{angle-func_iter5.png}
    \caption{Итерация 5}
  \end{subfigure}
  \begin{subfigure}{0.32\textwidth}
    \centering
    \includegraphics[width=\linewidth]{angle-func_iter10.png}
    \caption{Итерация 10}
  \end{subfigure}

  \begin{subfigure}{0.49\textwidth}
    \centering
    \includegraphics[width=\linewidth]{angle-func_iter20.png}
    \caption{Итерация 20}
  \end{subfigure}
  \begin{subfigure}{0.49\textwidth}
    \centering
    \includegraphics[width=\linewidth]{angle-func_iter25.png}
    \caption{Итерация 25}
  \end{subfigure}

  \caption{Эволюция ломаной (Пиявского--Шуберта) для \texttt{angle-func}.}
\end{figure}

\subsection{angular-velocity-function}
\begin{figure}[H]
  \centering
  \begin{subfigure}{0.32\textwidth}
    \centering
    \includegraphics[width=\linewidth]{angular-velocity-function_iter1.png}
    \caption{Итерация 1}
  \end{subfigure}
  \begin{subfigure}{0.32\textwidth}
    \centering
    \includegraphics[width=\linewidth]{angular-velocity-function_iter5.png}
    \caption{Итерация 5}
  \end{subfigure}
  \begin{subfigure}{0.32\textwidth}
    \centering
    \includegraphics[width=\linewidth]{angular-velocity-function_iter10.png}
    \caption{Итерация 10}
  \end{subfigure}

  \begin{subfigure}{0.49\textwidth}
    \centering
    \includegraphics[width=\linewidth]{angular-velocity-function_iter20.png}
    \caption{Итерация 20}
  \end{subfigure}
  \begin{subfigure}{0.49\textwidth}
    \centering
    \includegraphics[width=\linewidth]{angular-velocity-function_iter25.png}
    \caption{Итерация 25}
  \end{subfigure}

  \caption{Эволюция ломаной (Пиявского--Шуберта) для \texttt{angular-velocity-function}.}
\end{figure}

\subsection{distance\_txt}
\begin{figure}[H]
  \centering
  \begin{subfigure}{0.32\textwidth}
    \centering
    \includegraphics[width=\linewidth]{distance_txt_iter1.png}
    \caption{Итерация 1}
  \end{subfigure}
  \begin{subfigure}{0.32\textwidth}
    \centering
    \includegraphics[width=\linewidth]{distance_txt_iter5.png}
    \caption{Итерация 5}
  \end{subfigure}
  \begin{subfigure}{0.32\textwidth}
    \centering
    \includegraphics[width=\linewidth]{distance_txt_iter10.png}
    \caption{Итерация 10}
  \end{subfigure}

  \begin{subfigure}{0.49\textwidth}
    \centering
    \includegraphics[width=\linewidth]{distance_txt_iter20.png}
    \caption{Итерация 20}
  \end{subfigure}
  \begin{subfigure}{0.49\textwidth}
    \centering
    \includegraphics[width=\linewidth]{distance_txt_iter25.png}
    \caption{Итерация 25}
  \end{subfigure}

  \caption{Эволюция ломаной (Пиявского--Шуберта) для \texttt{distance\_txt}.}
\end{figure}

\subsection{distatnce\_csv (да, с опечаткой в имени файла)}
\begin{figure}[H]
  \centering
  \begin{subfigure}{0.32\textwidth}
    \centering
    \includegraphics[width=\linewidth]{distatnce_csv_iter1.png}
    \caption{Итерация 1}
  \end{subfigure}
  \begin{subfigure}{0.32\textwidth}
    \centering
    \includegraphics[width=\linewidth]{distatnce_csv_iter5.png}
    \caption{Итерация 5}
  \end{subfigure}
  \begin{subfigure}{0.32\textwidth}
    \centering
    \includegraphics[width=\linewidth]{distatnce_csv_iter10.png}
    \caption{Итерация 10}
  \end{subfigure}

  \begin{subfigure}{0.49\textwidth}
    \centering
    \includegraphics[width=\linewidth]{distatnce_csv_iter20.png}
    \caption{Итерация 20}
  \end{subfigure}
  \begin{subfigure}{0.49\textwidth}
    \centering
    \includegraphics[width=\linewidth]{distatnce_csv_iter25.png}
    \caption{Итерация 25}
  \end{subfigure}

  \caption{Эволюция ломаной (Пиявского--Шуберта) для \texttt{distatnce\_csv}.}
\end{figure}

\subsection{satellite-facility-dist}
\begin{figure}[H]
  \centering
  \begin{subfigure}{0.32\textwidth}
    \centering
    \includegraphics[width=\linewidth]{satellite-facility-dist_iter1.png}
    \caption{Итерация 1}
  \end{subfigure}
  \begin{subfigure}{0.32\textwidth}
    \centering
    \includegraphics[width=\linewidth]{satellite-facility-dist_iter5.png}
    \caption{Итерация 5}
  \end{subfigure}
  \begin{subfigure}{0.32\textwidth}
    \centering
    \includegraphics[width=\linewidth]{satellite-facility-dist_iter10.png}
    \caption{Итерация 10}
  \end{subfigure}

  \begin{subfigure}{0.49\textwidth}
    \centering
    \includegraphics[width=\linewidth]{satellite-facility-dist_iter20.png}
    \caption{Итерация 20}
  \end{subfigure}
  \begin{subfigure}{0.49\textwidth}
    \centering
    \includegraphics[width=\linewidth]{satellite-facility-dist_iter25.png}
    \caption{Итерация 25}
  \end{subfigure}

  \caption{Эволюция ломаной (Пиявского--Шуберта) для \texttt{satellite-facility-dist}.}
\end{figure}

\section{Критические замечания по реализации}

\subsection{Выбор интервала в методе ломаных}
По теории для минимизации нужно выбирать интервал с \textbf{минимальным} \(R_i\):
\[
i^\ast=\arg\min_i R_i
\]
Если выбрать \(\arg\max_i R_i\), то алгоритм систематически уточняет интервалы,
где нижняя оценка \emph{хуже} (выше), что противоречит логике глобального поиска минимума.

\subsection{Значение \(L\)}
Если \(L\) занижен, минорнат \(g(x)\) перестаёт быть нижней оценкой функции,
и гарантии глобальности теряются. Практически это лечится выбором \(r>1\) и/или увеличением \(r\) при подозрительных данных.

\subsection{Унимодальные методы}
Дихотомия/золотое сечение/Фибоначчи корректны и эффективны только при унимодальности.
На многомодальных функциях они находят локальный минимум, зависящий от формы \(\tilde f\).

\section{Вывод}
Описанный алгоритм объединяет:
\begin{itemize}
  \item корректную работу с табличными данными и построение \(\tilde f\) (линейная интерполяция);
  \item оценку липшицевой константы;
  \item глобальный метод Пиявского--Шуберта, способный искать глобальный минимум при корректном \(L\);
  \item унимодальные методы как быстрые локальные альтернативы/бенчмарки.
\end{itemize}

\end{document}