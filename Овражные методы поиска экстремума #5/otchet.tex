\documentclass[12pt,a4paper]{article}
\usepackage[T2A]{fontenc}
\usepackage[utf8]{inputenc}
\usepackage[russian]{babel}
\usepackage{geometry}
\geometry{margin=2.5cm}
\usepackage{amsmath}
\usepackage{graphicx, amsmath, amssymb, hyperref, placeins, cite}
\usepackage{float}
\hypersetup{hidelinks}
\usepackage{indentfirst}
\usepackage{array}
\usepackage{booktabs}

\begin{document}

\begin{center}
    \textbf{Отчёт по лабораторной работе}\\[0.5em]
    \textbf{Метод градиентного спуска}
\end{center}

\section{Постановка задачи}

\section{Постановка задачи}

Рассматривается модификация метода градиентного спуска, называемая овражным
методом, для минимизации функций
\begin{align*}
    J_1(x, y) &= x^2 + \alpha y^2,\quad \alpha \gg 1,\\
    J_2(x, y) &= \frac{x^2}{a^2} + \frac{y^2}{b^2},\quad a \gg b,\\
    J_3(x, y) &= 70 (x-1)^2 + (y-1)^2 + 1,\\
    J_4(x, y) &= (x-8)^2 + (y-1)^2 + 70 \bigl(y + (x-8)^2 - 1\bigr)^2 + 1,\\
    J_5(x, y) &= (x-8)^2 + (y-1)^2 + 70(x+y-9)^2 + 1.
\end{align*}

В численных экспериментах использовались значения $\alpha = 10$,
$a = 10$, $b = 1$, что соответствует сильно вытянутым линиям уровня
в задачах $J_1$ и $J_2$.

\section{Описание овражного метода}

Пусть задана два начальные точки $v_0$ и $v_1$ на плоскости. Из каждой
из них выполняется один шаг метода градиентного спуска с шагом
$h_{\text{gd}} > 0$:
\[
    x_0 = v_0 - h_{\text{gd}} \nabla J(v_0), \qquad
    x_1 = v_1 - h_{\text{gd}} \nabla J(v_1).
\]

Далее для $k \ge 1$ предполагается, что найдены точки
$x_{k-1}$ и $x_k$. Строится вспомогательная точка
\[
    v_k = x_k - \frac{x_k - x_{k-1}}{\|x_k - x_{k-1}\|}
    \, h \, \operatorname{sign}\bigl(J(x_k) - J(x_{k-1})\bigr),
\]
где $h > 0$~--- параметр овражного шага.
Затем из точки $v_k$ выполняется один шаг градиентного спуска:
\[
    x_{k+1} = v_k - h_{\text{gd}} \nabla J(v_k).
\]

Последовательность $\{x_k\}$ строится до выполнения одного из критериев
остановки:
\begin{itemize}
    \item ограничение на число итераций $k_{\max}$;
    \item малое изменение аргумента:
    \[
        \|x_{k+1} - x_k\| < \varepsilon_x;
    \]
    \item малое изменение функционала:
    \[
        |J(x_{k+1}) - J(x_k)| < \varepsilon_f.
    \]
\end{itemize}

В реализованном коде использовались значения
$\varepsilon_x = 10^{-6}$, $\varepsilon_f = 10^{-6}$,
$k_{\max} = 20000$.

\section{Численные эксперименты}

Для каждой из функций $J_1$--$J_5$ выбирались начальные точки $v_0$, $v_1$
и параметры шага $h_{\text{gd}}$ и $h$:

\begin{itemize}
    \item $J_1$: $v_0 = (5, 5)$, $v_1 = (4, 5)$,
    $h_{\text{gd}} = 0.05$, $h = 0.5$;
    \item $J_2$: $v_0 = (5, 3)$, $v_1 = (5, 4)$,
    $h_{\text{gd}} = 0.05$, $h = 0.5$;
    \item $J_3$: $v_0 = (-5, 5)$, $v_1 = (-4, 5)$,
    $h_{\text{gd}} = 0.01$, $h = 0.2$;
    \item $J_4$: $v_0 = (8, 3)$, $v_1 = (7, 3)$,
    $h_{\text{gd}} = 0.001$, $h = 0.1$;
    \item $J_5$: $v_0 = (8, 3)$, $v_1 = (9, 2)$,
    $h_{\text{gd}} = 0.005$, $h = 0.1$.
\end{itemize}


\subsection{Функция $J_1(x, y) = x^2 + 10 y^2$}

\subsubsection*{Траектория на плоскости}

\begin{figure}[H]
    \centering
    \includegraphics[width=0.7\textwidth]{j1_traj_ravine.png}
    \caption{Траектория овражного метода для $J_1$
    (файл \texttt{j1\_traj\_ravine.png}).}
    \label{fig:j1_traj_ravine}
\end{figure}

\subsubsection*{Убывание значения функционала $J_1(x_k)$}

\begin{figure}[H]
    \centering
    \includegraphics[width=0.7\textwidth]{j1_J_ravine.png}
    \caption{График убывания $J_1(x_k)$ при использовании овражного метода
    (файл \texttt{j1\_J\_ravine.png}).}
    \label{fig:j1_J_ravine}
\end{figure}

\subsection{Функция $J_2(x, y) = \frac{{x^2}}{{{10}^2}} + \frac{{y^2}}{{{1}^2}}$}

\subsubsection*{Траектории на плоскости}

\begin{figure}[H]
    \centering
    \includegraphics[width=0.7\textwidth]{j2_traj_ravine.png}
    \caption{Траектория овражного метода для $J_2$
    (файл \texttt{j2\_traj\_ravine.png}).}
    \label{fig:j2_traj_ravine}
\end{figure}

\begin{figure}[H]
    \centering
    \includegraphics[width=0.7\textwidth]{j2_J_ravine.png}
    \caption{График убывания $J_2(x_k)$ при использовании овражного метода
    (файл \texttt{j2\_J\_ravine.png}).}
    \label{fig:j2_J_ravine}
\end{figure}

\subsection{Функция $J_3(x, y) = 70(x-1)^2 + (y-1)^2 + 1$}

\subsubsection*{Траектории на плоскости}

\begin{figure}[H]
    \centering
    \includegraphics[width=0.7\textwidth]{j3_traj_ravine.png}
    \caption{Траектория овражного метода для $J_3$
    (файл \texttt{j3\_traj\_ravine.png}).}
    \label{fig:j3_traj_ravine}
\end{figure}

\begin{figure}[H]
    \centering
    \includegraphics[width=0.7\textwidth]{j3_J_ravine.png}
    \caption{График убывания $J_3(x_k)$ при использовании овражного метода
    (файл \texttt{j3\_J\_ravine.png}).}
    \label{fig:j3_J_ravine}
\end{figure}

\subsection{Функция $J_4(x, y) = (x-8)^2 + (y-1)^2 + 70(y+(x-8)^2-1)^2 + 1$}

\subsubsection*{Траектории на плоскости}

\begin{figure}[H]
    \centering
    \includegraphics[width=0.7\textwidth]{j4_traj_ravine.png}
    \caption{Траектория овражного метода для $J_4$
    (файл \texttt{j4\_traj\_ravine.png}).}
    \label{fig:j4_traj_ravine}
\end{figure}

\begin{figure}[H]
    \centering
    \includegraphics[width=0.7\textwidth]{j4_J_ravine.png}
    \caption{График убывания $J_4(x_k)$ при использовании овражного метода
    (файл \texttt{j4\_J\_ravine.png}).}
    \label{fig:j4_J_ravine}
\end{figure}

\subsection{Функция $J_5(x, y) = (x-8)^2 + (y-1)^2 + 70(x+y-9)^2 + 1$}

\subsubsection*{Траектории на плоскости}

\begin{figure}[H]
    \centering
    \includegraphics[width=0.7\textwidth]{j5_traj_ravine.png}
    \caption{Траектория овражного метода для $J_5$
    (файл \texttt{j5\_traj\_ravine.png}).}
    \label{fig:j5_traj_ravine}
\end{figure}

\begin{figure}[H]
    \centering
    \includegraphics[width=0.7\textwidth]{j5_J_ravine.png}
    \caption{График убывания $J_5(x_k)$ при использовании овражного метода
    (файл \texttt{j5\_J\_ravine.png}).}
    \label{fig:j5_J_ravine}
\end{figure}

\section{Итоговые точки минимума}

Результаты работы метода сведены в таблицу

% \begin{table}[H]
%     \centering
%     \begin{tabular}{|c|c|c|c|c|}
%         \hline
%         Функция & Режим шага & Найденная точка $x^\ast$ & $J(x^\ast)$ & Критерий остановки \\
%         \hline
%         $J_1$ & постоянный    & $(\ldots,\ \ldots)$ & $\ldots$ & $\ldots$ \\
%         $J_1$ & убывающий     & $(\ldots,\ \ldots)$ & $\ldots$ & $\ldots$ \\
%         $J_1$ & $h_k = 0.5/(k+1)$ & $(\ldots,\ \ldots)$ & $\ldots$ & $\ldots$ \\
%         $J_2$ & постоянный    & $(\ldots,\ \ldots)$ & $\ldots$ & $\ldots$ \\
%         $J_2$ & убывающий     & $(\ldots,\ \ldots)$ & $\ldots$ & $\ldots$ \\
%         $J_3$ & постоянный    & $(\ldots,\ \ldots)$ & $\ldots$ & $\ldots$ \\
%         $J_3$ & убывающий     & $(\ldots,\ \ldots)$ & $\ldots$ & $\ldots$ \\
%         $J_4$ & постоянный    & $(\ldots,\ \ldots)$ & $\ldots$ & $\ldots$ \\
%         $J_4$ & убывающий     & $(\ldots,\ \ldots)$ & $\ldots$ & $\ldots$ \\
%         \hline
%     \end{tabular}
%     \caption{Итоговые точки минимума и значения функционала для разных режимов шага.}
%     \label{tab:results_summary}
% \end{table}
\input{ravine_results_table.tex}

\section{Выводы}

По результатам численных экспериментов с овражным методом можно сделать
следующие выводы.

\begin{enumerate}
    \item Для всех рассмотренных функций $J_1$--$J_5$ овражный метод
    сходится к ожидаемым точкам минимума: $(0,0)$ для $J_1$ и $J_2$,
    $(1,1)$ для $J_3$, $(8,1)$ для $J_4$ и $(8,1)$ для $J_5$.
    Численные значения $x^*$ и $J(x^*)$, приведённые в таблице,
    совпадают с теоретическими значениями с погрешностью порядка
    $10^{-3}$--$10^{-6}$.

    \item На сильно вытянутых квадратичных функциях ($J_1$, $J_2$)
    овражный шаг заметно уменьшает “зигзагообразное” движение, характерное
    для обычного градиентного спуска: траектория становится более
    направленной вдоль долины, а убывание $J(x_k)$ происходит более равномерно.
    При этом слишком большие значения параметра $h$ приводят к переусилению
    коррекции и могут ухудшить сходимость.

    \item В задаче $J_3$ с резким различием кривизны по $x$ и $y$ овражный
    метод позволяет использовать более агрессивный шаг градиентного
    спуска $h_{\text{gd}}$, чем в чистом градиентном методе, оставаясь
    при этом устойчивым. Овражный шаг частично компенсирует дрейф
    поперёк узкой долины и ускоряет выход на окрестность минимума.

    \item Для сильно нелинейной функции $J_4$ с узкой искривлённой долиной
    овражный метод также уменьшает “перескакивание” через долину, однако
    чувствителен к выбору параметров: при слишком большом $h$ возможны
    колебания вдоль долины, при слишком малом эффект овражной коррекции почти
    исчезает и метод приближается к обычному градиентному спуску.

    \item В задаче $J_5$, где долина ориентирована вдоль наклонной прямой
    $x + y = 9$, овражный метод эффективно учитывает изменение направления
    долины по ходу итераций: траектория быстрее выстраивается вдоль линии
    минимума, а число шагов до заданной точности уменьшается по сравнению
    с “наивным” градиентным спуском.

    \item В целом овражный метод можно рассматривать как простую и дешёвую
    по вычислениям модификацию градиентного спуска, которая улучшает
    поведение метода в сильно вытянутых и “овражных” задачах, при этом
    требуя аккуратного подбора двух параметров: шага градиентного спуска
    $h_{\text{gd}}$ и овражного шага $h$.
\end{enumerate}

\end{document}