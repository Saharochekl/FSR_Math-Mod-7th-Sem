\documentclass[12pt,a4paper]{article}
\usepackage[T2A]{fontenc}
\usepackage[utf8]{inputenc}
\usepackage[russian]{babel}
\usepackage{geometry}
\geometry{margin=2.5cm}
\usepackage{amsmath}
\usepackage{graphicx, amsmath, amssymb, hyperref, placeins, cite}
\usepackage{float}
\hypersetup{hidelinks}
\usepackage{indentfirst}
\usepackage{array}
\usepackage{booktabs}

\begin{document}

\begin{center}
    \textbf{Отчёт по лабораторной работе}\\[0.5em]
    \textbf{Метод градиентного спуска}
\end{center}

\section{Постановка задачи}

В работе реализуется метод градиентного спуска для минимизации следующих функций:
\begin{align*}
    J_1(x, y) &= x^2 + y^2,\\
    J_2(x, y) &= \frac{x^2}{a^2} + \frac{y^2}{b^2},\\
    J_3(x, y) &= 70 (x-1)^2 + (y-1)^2 + 1,\\
    J_4(x, y) &= (x-8)^2 + (y-1)^2 + 70 \bigl(y + (x-8)^2 - 1\bigr)^2 + 1,
\end{align*}
где параметры $a, b$ задают большую и малую полуоси эллипса во второй функции.

Метод градиентного спуска настраивается по:
\begin{itemize}
    \item начальной точке $x_0 = (x_0, y_0)$;
    \item режиму выбора шага $h_k$;
    \item значению шага $h$ (или параметрам $c, \alpha$ для убывающих шагов);
    \item критерию остановки: по числу итераций, по норме разности векторов $\|x_{k+1} - x_k\|$, по изменению функционала $|J(x_{k+1}) - J(x_k)|$.
\end{itemize}

\section{Описание метода}

На $k$-й итерации метода градиентного спуска вычисляется шаг
\[
    x_{k+1} = x_k - h_k \nabla J(x_k),
\]
где $h_k$ выбирается в одном из следующих режимов:
\begin{itemize}
    \item постоянный шаг: $h_k = h$;
    \item убывающий шаг вида
    \[
        h_k = c (1 + k)^{-\alpha}, \quad c > 0,\; 0.5 < \alpha \leq 1;
    \]
    \item шаг вида
    \[
        h_k = \frac{c}{k+1}, \quad c > 0,
    \]
    удовлетворяющий условиям
    \[
        h_k > 0,\quad \sum_{k=0}^{\infty} h_k = \infty,\quad
        \sum_{k=0}^{\infty} h_k^2 < \infty.
    \]
\end{itemize}

Критерии остановки:
\begin{itemize}
    \item ограничение сверху на число итераций $k_{\max}$;
    \item достижение малой нормы приращения аргумента:
    \[
        \|x_{k+1} - x_k\| < \varepsilon_x;
    \]
    \item достижение малой разности значений функционала:
    \[
        |J(x_{k+1}) - J(x_k)| < \varepsilon_f.
    \]
\end{itemize}

В реализованном коде используются значения
\[
    \varepsilon_x = 10^{-6}, \quad \varepsilon_f = 10^{-6},
\]
а максимальное число итераций выбирается равным $k_{\max} = 20000$.

\section{Численные эксперименты}

Для каждой функции рассматриваются два режима шага:
\begin{itemize}
    \item постоянный шаг ($h_k = h$);
    \item убывающий шаг вида $h_k = c (1 + k)^{-\alpha}$.
\end{itemize}
Отдельно демонстрируется режим $h_k = c/(k+1)$ на функции $J_1$.

Начальные точки и значения шагов:

В численных экспериментах для функции $J_2$ использовались значения
$a = 5$, $b = 1$, что соответствует вытянутому эллипсу уровней вдоль оси $x$.
\begin{itemize}
    \item $J_1$: $x_0 = (5, 5)$, постоянный шаг $h = 0.2$, убывающий шаг $h_k = 0.5 (1 + k)^{-0.6}$;
    \item $J_2$: $x_0 = (5, 3)$, постоянный шаг $h = 0.2$, убывающий шаг $h_k = 0.5 (1 + k)^{-0.6}$;
    \item $J_3$: $x_0 = (-5, 5)$, постоянный шаг $h = 0.01$, убывающий шаг $h_k = 0.5 (1 + k)^{-0.9}$;
    \item $J_4$: $x_0 = (8, 3)$, постоянный шаг $h = 0.001$, убывающий шаг $h_k = 0.05 (1 + k)^{-0.9}$;
    \item демонстрация шага $h_k = \dfrac{0.5}{k+1}$ для $J_1$.
\end{itemize}

\subsection{Функция $J_1(x, y) = x^2 + y^2$}

\subsubsection*{Траектории на плоскости}

\begin{figure}[H]
    \centering
    \includegraphics[width=0.7\textwidth]{f1_traj_constant.png}
    \caption{Траектория градиентного спуска для $J_1$ при постоянном шаге
    (файл \texttt{f1\_traj\_constant.png}).}
    \label{fig:f1_traj_constant}
\end{figure}

\begin{figure}[H]
    \centering
    \includegraphics[width=0.7\textwidth]{f1_traj_power.png}
    \caption{Траектория градиентного спуска для $J_1$ при убывающем шаге
    $h_k = 0.5 (1 + k)^{-0.6}$ (файл \texttt{f1\_traj\_power.png}).}
    \label{fig:f1_traj_power}
\end{figure}

Если используется демонстрация режима $h_k = 0.5/(k+1)$, то соответствующий график траектории:
\begin{figure}[H]
    \centering
    \includegraphics[width=0.7\textwidth]{f1_traj_1_over_k.png}
    \caption{Траектория градиентного спуска для $J_1$ при шаге
    $h_k = 0.5/(k+1)$ (файл \texttt{f1\_traj\_1\_over\_k.png}).}
    \label{fig:f1_traj_1_over_k}
\end{figure}

\subsubsection*{Убывание значения функционала $J_1(x_k)$}

\begin{figure}[H]
    \centering
    \includegraphics[width=0.7\textwidth]{f1_J_constant.png}
    \caption{График убывания $J_1(x_k)$ при постоянном шаге
    (файл \texttt{f1\_J\_constant.png}).}
    \label{fig:f1_J_constant}
\end{figure}

\begin{figure}[H]
    \centering
    \includegraphics[width=0.7\textwidth]{f1_J_power.png}
    \caption{График убывания $J_1(x_k)$ при убывающем шаге
    $h_k = 0.5 (1 + k)^{-0.6}$ (файл \texttt{f1\_J\_power.png}).}
    \label{fig:f1_J_power}
\end{figure}

При необходимости:
\begin{figure}[H]
    \centering
    \includegraphics[width=0.7\textwidth]{f1_J_1_over_k.png}
    \caption{График убывания $J_1(x_k)$ при шаге
    $h_k = 0.5/(k+1)$ (файл \texttt{f1\_J\_1\_over\_k.png}).}
    \label{fig:f1_J_1_over_k}
\end{figure}

\subsection{Функция $J_2(x, y) = \dfrac{x^2}{a^2} + \dfrac{y^2}{b^2}$}

\subsubsection*{Траектории на плоскости}

\begin{figure}[H]
    \centering
    \includegraphics[width=0.7\textwidth]{f2_traj_constant.png}
    \caption{Траектория градиентного спуска для $J_2$ при постоянном шаге
    (файл \texttt{f2\_traj\_constant.png}).}
    \label{fig:f2_traj_constant}
\end{figure}

\begin{figure}[H]
    \centering
    \includegraphics[width=0.7\textwidth]{f2_traj_power.png}
    \caption{Траектория градиентного спуска для $J_2$ при убывающем шаге
    $h_k = 0.5 (1 + k)^{-0.6}$ (файл \texttt{f2\_traj\_power.png}).}
    \label{fig:f2_traj_power}
\end{figure}

\subsubsection*{Убывание $J_2(x_k)$}

\begin{figure}[H]
    \centering
    \includegraphics[width=0.7\textwidth]{f2_J_constant.png}
    \caption{График убывания $J_2(x_k)$ при постоянном шаге
    (файл \texttt{f2\_J\_constant.png}).}
    \label{fig:f2_J_constant}
\end{figure}

\begin{figure}[H]
    \centering
    \includegraphics[width=0.7\textwidth]{f2_J_power.png}
    \caption{График убывания $J_2(x_k)$ при убывающем шаге
    (файл \texttt{f2\_J\_power.png}).}
    \label{fig:f2_J_power}
\end{figure}

\subsection{Функция $J_3(x, y) = 70 (x-1)^2 + (y-1)^2 + 1$}

\subsubsection*{Траектории на плоскости}

\begin{figure}[H]
    \centering
    \includegraphics[width=0.7\textwidth]{f3_traj_constant.png}
    \caption{Траектория градиентного спуска для $J_3$ при постоянном шаге
    (файл \texttt{f3\_traj\_constant.png}).}
    \label{fig:f3_traj_constant}
\end{figure}

\begin{figure}[H]
    \centering
    \includegraphics[width=0.7\textwidth]{f3_traj_power.png}
    \caption{Траектория градиентного спуска для $J_3$ при убывающем шаге
    $h_k = 0.5 (1 + k)^{-0.9}$ (файл \texttt{f3\_traj\_power.png}).}
    \label{fig:f3_traj_power}
\end{figure}

\subsubsection*{Убывание $J_3(x_k)$}

\begin{figure}[H]
    \centering
    \includegraphics[width=0.7\textwidth]{f3_J_constant.png}
    \caption{График убывания $J_3(x_k)$ при постоянном шаге
    (файл \texttt{f3\_J\_constant.png}).}
    \label{fig:f3_J_constant}
\end{figure}

\begin{figure}[H]
    \centering
    \includegraphics[width=0.7\textwidth]{f3_J_power.png}
    \caption{График убывания $J_3(x_k)$ при убывающем шаге
    (файл \texttt{f3\_J\_power.png}).}
    \label{fig:f3_J_power}
\end{figure}

\subsection{Функция $J_4(x, y)$}

\subsubsection*{Траектории на плоскости}

\begin{figure}[H]
    \centering
    \includegraphics[width=0.7\textwidth]{f4_traj_constant.png}
    \caption{Траектория градиентного спуска для $J_4$ при постоянном шаге
    (файл \texttt{f4\_traj\_constant.png}).}
    \label{fig:f4_traj_constant}
\end{figure}

\begin{figure}[H]
    \centering
    \includegraphics[width=0.7\textwidth]{f4_traj_power.png}
    \caption{Траектория градиентного спуска для $J_4$ при убывающем шаге
    $h_k = 0.05 (1 + k)^{-0.9}$ (файл \texttt{f4\_traj\_power.png}).}
    \label{fig:f4_traj_power}
\end{figure}

\subsubsection*{Убывание $J_4(x_k)$}

\begin{figure}[H]
    \centering
    \includegraphics[width=0.7\textwidth]{f4_J_constant.png}
    \caption{График убывания $J_4(x_k)$ при постоянном шаге
    (файл \texttt{f4\_J\_constant.png}).}
    \label{fig:f4_J_constant}
\end{figure}

\begin{figure}[H]
    \centering
    \includegraphics[width=0.7\textwidth]{f4_J_power.png}
    \caption{График убывания $J_4(x_k)$ при убывающем шаге
    (файл \texttt{f4\_J\_power.png}).}
    \label{fig:f4_J_power}
\end{figure}

\section{Итоговые точки минимума}

Результаты работы метода сведены в таблицу

% \begin{table}[H]
%     \centering
%     \begin{tabular}{|c|c|c|c|c|}
%         \hline
%         Функция & Режим шага & Найденная точка $x^\ast$ & $J(x^\ast)$ & Критерий остановки \\
%         \hline
%         $J_1$ & постоянный    & $(\ldots,\ \ldots)$ & $\ldots$ & $\ldots$ \\
%         $J_1$ & убывающий     & $(\ldots,\ \ldots)$ & $\ldots$ & $\ldots$ \\
%         $J_1$ & $h_k = 0.5/(k+1)$ & $(\ldots,\ \ldots)$ & $\ldots$ & $\ldots$ \\
%         $J_2$ & постоянный    & $(\ldots,\ \ldots)$ & $\ldots$ & $\ldots$ \\
%         $J_2$ & убывающий     & $(\ldots,\ \ldots)$ & $\ldots$ & $\ldots$ \\
%         $J_3$ & постоянный    & $(\ldots,\ \ldots)$ & $\ldots$ & $\ldots$ \\
%         $J_3$ & убывающий     & $(\ldots,\ \ldots)$ & $\ldots$ & $\ldots$ \\
%         $J_4$ & постоянный    & $(\ldots,\ \ldots)$ & $\ldots$ & $\ldots$ \\
%         $J_4$ & убывающий     & $(\ldots,\ \ldots)$ & $\ldots$ & $\ldots$ \\
%         \hline
%     \end{tabular}
%     \caption{Итоговые точки минимума и значения функционала для разных режимов шага.}
%     \label{tab:results_summary}
% \end{table}
\begin{table}[H]
    \centering
\begin{tabular}{|c|c|c|c|c|}
\toprule
Функция & Режим шага & Найденная точка $x^*$ & $J(x^*)$ & Критерий остановки \\
\midrule
$J_1$ & постоянный шаг h = 0.2 & (0.000508, 0.000508) & 0.00000052 & tol_f: |J_(k+1)-J_k| = 9.17e-07 \\
$J_1$ & $h_k = 0.5 (1 + k)^{-0.6}$ & (0.000000, 0.000000) & 0.00000000 & tol_x: ||x_(k+1)-x_k|| = 0.00e+00 \\
$J_2$ & постоянный шаг h = 0.2 & (0.027314, 0.000000) & 0.00002984 & tol_f: |J_(k+1)-J_k| = 9.78e-07 \\
$J_2$ & $h_k = 0.5 (1 + k)^{-0.6}$ & (0.234139, 0.000000) & 0.00219284 & tol_f: |J_(k+1)-J_k| = 1.00e-06 \\
$J_3$ & постоянный шаг h = 0.01 & (1.000000, 1.004888) & 1.00002389 & tol_f: |J_(k+1)-J_k| = 9.85e-07 \\
$J_3$ & $h_k = 0.5 (1 + k)^{-0.9}$ & (1.000000, 1.000000) & 1.00000000 & tol_f: |J_(k+1)-J_k| = 1.90e-08 \\
$J_4$ & постоянный шаг h = 0.001 & (8.000000, 1.000175) & 1.00000218 & tol_f: |J_(k+1)-J_k| = 7.82e-07 \\
$J_4$ & $h_k = 0.05 (1 + k)^{-0.9}$ & (8.000000, 1.000028) & 1.00000005 & tol_f: |J_(k+1)-J_k| = 8.79e-07 \\
$J_1$ & h_k = 0.5 / (k+1) & (0.000000, 0.000000) & 0.00000000 & tol_x: ||x_(k+1)-x_k|| = 0.00e+00 \\
\bottomrule
\end{tabular}

    \caption{Итоговые точки минимума и значения функционала для разных режимов шага.}
    \label{tab:results_summary}
\end{table}


\section{Выводы}

По результатам численных экспериментов можно сделать следующие выводы.

\begin{enumerate}
    \item Для всех рассмотренных функций $J_1$--$J_4$ метод градиентного спуска
    сходится к аналитически известным точкам минимума:
    $J_1$ и $J_2$~--- к точке $(0, 0)$, $J_3$~--- к точке $(1, 1)$,
    $J_4$~--- к точке $(8, 1)$. Численные значения $x^\ast$ и $J(x^\ast)$,
    приведённые в таблице~\ref{tab:results_summary}, совпадают с теоретическими
    с погрешностью порядка $10^{-3}$--$10^{-6}$.
    
    \item На хорошо обусловленной квадратичной функции $J_1$ постоянный шаг и
    убывающий шаг демонстрируют быструю и стабильную сходимость. При разумном
    выборе $h$ постоянный шаг обеспечивает почти геометрическое убывание $J(x_k)$.
    Шаг вида $h_k = c(1+k)^{-\alpha}$ даёт более плавную, но гарантированно
    устойчивую сходимость.
    
    \item Для вытянутой квадратичной функции $J_2$ влияние обусловленности
    заметно сильнее: траектория с постоянным шагом существенно искривлена,
    и сходится медленнее вдоль “длинной” оси эллипса. Переход к убывающему шагу
    частично сглаживает колебания и делает поведение метода более устойчивым,
    хотя скорость сходимости на начальных итерациях снижается.
    
    \item Для функции $J_3$ с сильно различающимися коэффициентами по $x$ и $y$
    (жёсткое направление по $x$) требуются малые значения постоянного шага,
    иначе возникает риск расходимости. Убывающий шаг позволяет выбрать более
    агрессивное начальное значение и при этом гарантировать сходимость за счёт
    постепенного уменьшения $h_k$.
    
    \item Для нелинейной функции $J_4$ с узкой искривлённой долиной градиентный
    спуск чувствителен к выбору шага: слишком большие значения $h$ приводят к
    “прыжкам” поперёк долины и замедлению сходимости. Правильно подобранный
    постоянный шаг и убывающий шаг обеспечивают движение вдоль долины к минимуму,
    что видно на траекториях и графиках убывания $J(x_k)$.
    
    \item Шаг вида $h_k = \dfrac{c}{k+1}$ на примере функции $J_1$ демонстрирует
    теоретически корректное поведение: условия $\sum h_k = \infty$ и
    $\sum h_k^2 < \infty$ выполняются, метод сходится к минимуму, однако убывание
    значения функционала происходит медленнее, чем при оптимально подобранном
    постоянном шаге.
\end{enumerate}

\end{document}